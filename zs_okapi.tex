\subsection{Retrieval Method}

Um unsere Methode zu testen, müssen wir uns zunächst auf eine retrieval method bzw. Ranking Methode festlegen. Die Herren Na und Ng verwenden hierfür eine Variante des BM25-Models, welche die CEEF-Methode zur Bestimmung der Termfrequency benutzt und wie folgt definiert ist:
\[ BM25_{CEEF}(e_Q;d) = OkapiTF(e_Q;d) \cdot IDF(e_Q) \]

wobei

\[ OkapiTF(e_Q;d) = \frac{ (k_1 + 1) \cdot tf_{CEEF}(e_Q;d)}{tf_{CEEF}(e_Q;d) + k_1\left( (1-b) + b \frac{len(d)}{avglen} \right)} \]

$b$ und $k_1$ dienen hier der Normierung. $IDF(e_Q)$ steht für die inverse Documentfrequency und wird vom BM25-Model zur Gewichtung verwendet und wird wie folgt berechnet:
\[ IDF(e_Q) = \log \left( \frac{N}{df(e_Q)} \right) \text{\cite{wiki:tf_idf}} \]

Es ist zudem noch zu erwähnen, dass diese Methode mit anderen aktuellen und als „state-of-the-art“ geltenden retrieval methods kombinierbar ist.\\
Dies funktioniert so, dass man sich einen Normierungfaktor $\alpha \in [0,1]$ wählt und damit die zwei Methoden entsprechend gewichtet und aufaddiert. Man nennt dieses Verfahren „Lineare Interpolation“ \cite{paper:Vogt}. 

\[ Sim(Q;d) = (1-\alpha) \cdot Sim_{BASE}(Q;d) + \alpha \cdot BM25_{CEEF}(e_Q;d) \]

Hier steht $Sim_{BASE}(Q;d)$ für ein beliebiges state-of-the-art Verfahren, zum Beispiel das Language Modeling \cite{paper:Zhai1}\cite{paper:Zhai2}.
