Zusammenfassend kann man sagen, dass Seung-Hoon Na und Hwee Tou Ng eine vielversprechende Methode gefunden haben, um die zeitaufwendigen und rechenlastigen vollständigen Auflösungung der Korefenzen zu umgehen.\\
Ihr Verfahren konnte in diversen Tests signifikant bessere Resultate liefern, verglichen mit anderen state-of-the-art retrieval Methoden, welche standard term frequency Schätzungen verwendeten. Es wurde sogar eine Verbesserung von 3\% gegenüber des Mean Average Precission Verfahren erziehlt.\\
Man sollte hier natürlich beachten, dass die Dokumente in ihren Tests stark ihren Bedürfnissen angepasst wurden. Zudem ließen sie es sich offen noch diversen Parameter zu tunen, damit die Tests unter entsprechenden Bedingungen, bessere Ergebnisse lieferten. Besonders auffällig hierfür sind die Normierungsparameter $k_1, b \text{ und } \alpha$ in ihrem BM25-Modell oder das $K$, welches man nach Belieben anpassen darf um $P(e_Q| A,d)$ zu berechnen.\\
Zudem konnten wir dem gesamten Paper keine Informationen über die tatsächliche Laufzeit ihrer Methode entnehmen. Es wurde zwar mehrfach erwähnt, dass \textbf{CEEF-FB-2POISSON} deutlich schneller als die vollständige Auflösung ist, aber wieviel schneller das Ganze wirklich ist, erfährt man leider nicht.\\

