\subsection{Eliteness}

\begin{defi}
	A document is \underline{elite} for a term if the document is ``about'' the concept represented by the term \cite{paper:Robertson}.
\end{defi}


Bei genauerer Begutachtung der Definition liegt die Annahme nahe, dass die Bezüge der anaphorischen Ausdrücke auf die Entitäten im Zusammenhang stehen, ob ein Dokument bezüglich dieser Entität „elite“ ist, oder nicht. Wir definieren uns also einen weiteren Faktor:

\begin{defi}
$P(\textbf{E}(e) = 1 | d)$ ist die Wahrscheinlichkeit, dass ein Dokument für eine Entität $e$ „elite“ ist.
\end{defi}

Jetzt können wir folgendes setzen:

\[ P(e_Q | A,d) = \frac{P(\textbf{E}(e) = 1 | d)}{\sum_{e \in \varepsilon (A;d)} P(\textbf{E}(e) = 1 | d)} \]

Wir machen also nichts anderes, als die Wahrscheinlichkeit der Eliteness des Dokumentes bezogen auf unsere gesuchte Entität in Relation mit den Wahrscheinlichkeiten der anderen Named Entities zu stellen.\\
Jetzt vereinfachen wir die Formel noch ein wenig, indem wir zuerst alle non-query Entitäten nach der Wahrscheinlichkeit der Eliteness sortieren. Danach werden die $K$ non-query Entitäten mit der höchsten Wahrscheinlichkeit selektiert, sodass sich unsere Formel auf Folgendes beschränkt:

\[ P(e_Q | A,d) \approx \frac{P(\textbf{E}(e) = 1 | d)}{P(\textbf{E}(e) = 1 | d) + \sum_{i=1}^{K} P(\textbf{E}(e_{n}^{(i)}) = 1 | d)} \]

$K$ ist in dem Fall frei gewählt und kann entsprechend angepasst werden. So dass tatsächlich nur Wahrscheinlichkeiten vernachlässigt werden, die nicht ins Gewicht fallen.\\
Nun suchen wir uns eine repräsentative non-query Entität heraus und setzen die Wahrscheinlichkeit der Eliteness aller anderen $K$ Entitäten mit der Wahrscheinlichkeit der Herausgesuchten gleich. Somit erhalten wir eine stark vereinfachte, aber trotzdem recht genaue Alternative der obigen Formel.

\[ P(e_Q | A,d) \approx \frac{P(\textbf{E}(e) = 1 | d)}{P(\textbf{E}(e) = 1 | d) + K \cdot P(\textbf{E}(e_N) = 1 | d)} \]


Im Folgenden werden wir eine performante und einfache Methode beschreiben, mit der es möglich ist, den Faktor $P(\textbf{E}(e) = 1 | d)$ zu berechnen.

