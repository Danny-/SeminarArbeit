\subsection{Formalisierung}
Bevor wir zur eigentlichen Statistik kommen, legen wir noch die mathematische Schreibweise für diese Aufgabe fest.
Zunächst sei Q unsere query. Die Query besteht hier zur Vereinfachung aus einem einzigen Eigennamen.\\
Weiter sei e ein beliebiger Eigennamen. Entspricht e der Suchanfrage der Query Q, so nennen wir den Eigennamen gesuchte Entität $e_Q$. Ist e ein beliebiger nicht gesuchter Eigennamen, so nennen wir die Entität nicht gesuchter Eigennamen $e_N$.\\


 \begin{itemize}
	\item $Q$ = query (Suche)
	\item $e_Q$ = query entity (gesuchte Entität)
	\item $e_N$ = non-query entity
	\item $A$ = Menge plausibler anaphorischer Ausdrücke
	\item $tf(A;d)$ = Anzahl von $A$ in Dokument $d$
	\item $\varepsilon (A;d)$ = Menge plausibler Entitäten in Dokument $d$
	\item $tf(e;d)$ = raw entity frequency
	\item $atf(e;A,d)$ = anaphoric entity frequency
  \end{itemize}
