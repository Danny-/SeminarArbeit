\subsection{Koreferenzen}

\begin{defi}
	\underline{Koreferenz} liegt vor, wenn in einer Äußerung mit zwei verschiedenen sprachlichen Ausdrücken dasselbe bezeichnet wird. \cite{wiki:Koreferenz}
\end{defi}

\fbox{ 
\begin{minipage}{12cm}
	\textbf{Alfred} ist eine Fantasiefigur. \underline{Er} entspringt einem illustrierenden Gedanken \ldots
\end{minipage}
}\\
\\
Im obigen Beispiel sehen wir zunächst einen Satz, der mit einem Eigennamen ``\textbf{Alfred}'' beginnt. Im nächsten Satz referenziert ``\underline{er}'' ebenfalls Alfred. Einen solchen Ausdruck, der einen zuvor genannten Inhalt wieder aufgreift, nennen wir anaphorischen Ausdruck.\\
\fbox{ 
\begin{minipage}{12cm}
	\textbf{Er} ist eine Fantasiefigur!!! \underline{Alfred}, die zunächst unerwähnte, mysteriöse Person, entspringt einem illustrierenden Gedanken \ldots
\end{minipage}
}\\
\\
Hier bezieht sich ``\textbf{er}'' auf den folgenden Eigennamen ``\underline{Alfred}''. Ein solches Gebilde nennt sich Katapher. Da wir im folgenden einen statistischen Ansatz verfolgen, der die Häufigkeit von Eigennamen mit den kataphorischen und anaphorischen Ausdrücken schätzen soll, sprechen wir in Zukunft nur noch von Koreferenzen.
