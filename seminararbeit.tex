% This template is provided for all the participants of the seminar ``Text Mining''
%%%%%%%%%%%%%%%%%%%%%
% Author information:
%%%%%%%%%%%%%%%%%%%%%
% Jannik Strötgen
% Institute of Computer Science
% Database Systems Research Group
% INF 348 
% 69120 Heidelberg
% stroetgen@uni-hd.de
%%%%%%%
% Date: April 24, 2010
%%%%%%% 

\documentclass[
     11pt,         % font size
     a4paper,      % paper format
     oneside,
     ]{article}

%%%%%%%%%%%%%%%%%%%%%%%%%%%%%%%%%%%%%%%%%%%%%%%%%%%%%%%%%%%%

% PACKAGES:

% Use German :
\usepackage[USenglish, ngerman]{babel}
% Input encoding
\usepackage[utf8]{inputenc}
% Font encoding
\usepackage[T1]{fontenc}
% Einbinden von URLs:
\usepackage{url}
% Include Graphic-files:
\usepackage{graphicx}
% Include PDF links
%\usepackage[pdftex, bookmarks=true]{hyperref}
% Fuer Textsatz
\usepackage{setspace}
% For bibliography style
\usepackage[numbers]{natbib}
% for Latex symbols
\usepackage{doc}
%amsmath
\usepackage{amsmath}
\usepackage{amsthm}

%%%%%%%%%%%%%%%%%%%%%%%%%%%%%%%%%%%%%%%%%%%
% Titel, Autor, Seminar, Semester, Dozent %
%%%%%%%%%%%%%%%%%%%%%%%%%%%%%%%%%%%%%%%%%%%
\newcommand{\mytitle}{Coreference of Named Entities}
\newcommand{\myauthor}{Philipp Schäfer, Daniel Kruck}
\newcommand{\myseminar}{Text Mining}
\newcommand{\mysemester}{Sommersemester 2010}
\newcommand{\mydozent}{Jannik Strötgen}
\newcommand{\mydozentTwo}{Prof. Dr. Michael Gertz}



\newtheoremstyle{custm}% name of the style to be used
  {}% measure of space to leave above the theorem. E.g.: 3pt
  {}% measure of space to leave below the theorem. E.g.: 3pt
  {}% name of font to use in the body of the theorem
  {0em}% measure of space to indent
  {\bf }% name of head font
  {}% punctuation between head and body
  {0.5cm}% space after theorem head; " " = normal interword space
  {}% Manually specify head

\theoremstyle{custm}

\newtheorem*{thm_defi}{Definition:}
\newenvironment{defi} [1][]{\begin{thm_defi} {\underline{\textbf{#1}}\ \\}}{\end{thm_defi} }


% OTHER SETTINGS:
\setlength{\parindent}{0in}

% Pagestyle:
\pagestyle{myheadings}
\markright{\myauthor: \mytitle}

\begin{document}

%%%%%%%%%%%%%%%%%%%%%%%%%%%%%%%%%%%%% <TITLE> %%%%%%%%%%%%%%%%%%%%%%%%%%%%%%%%%%%%%
\pagenumbering{roman}
\begin{titlepage}
\begin{tabular}[l]{l}
% Angaben zum Seminar
Ruprecht-Karls-Universität Heidelberg\\
Institut für Informatik\\
\mysemester\\
Seminar: \myseminar\\
Dozenten: \mydozent\\
\phantom{Dozenten: }\mydozentTwo\\
\end{tabular}

\vspace{4cm}
\begin{center}
\textbf{\large Seminararbeit} % Proseminararbeit,Studienarbeit, Interdisziplinaeres Projekt
\vspace{0.5\baselineskip}

% Titel wird ausgegeben (siehe oben)
{\huge
\mytitle
}
\end{center}

\vfill 
% Persönliche Angaben
\begin{tabular}[l]{ll}
Name:           & \myauthor\\
Matrikelnummer: & 2612579 (Philipp), 2440234 (Daniel)\\
Studiengang:    & Angewandte Informatik (6. Semester)\\
Email: & trashzopf@googlemail.com (Philipp), daniel.kruck@gmx.net (Daniel)\\
Datum der Abgabe: & \today \\
\end{tabular}

\end{titlepage}
%%%%%%%%%%%%%%%%%%%%%%%%%%%%%%%%%%%%% </TITLE> %%%%%%%%%%%%%%%%%%%%%%%%%%%%%%%%%%%%

% Zeilenabstand
\onehalfspacing


%%%%%%%%%%%%%%%%%%%%%%%%%%%%% <Antiplagiatserklärung> %%%%%%%%%%%%%%%%%%%%%%%%%%%%%
\thispagestyle{empty}
\vspace*{100pt}
Hiermit versichere ich \textbf{\myauthor}, dass ich die Hausarbeit mit dem Titel \textbf{\mytitle}
im Seminar \textbf{\myseminar}
im \textbf{\mysemester}
bei \textbf{\mydozent} und \textbf{\mydozentTwo}
selbstständig und nur mit den in der Arbeit angegebenen Hilfsmitteln verfasst habe.
Zitate sowie der Gebrauch fremder Quellen, Texte und Hilfsmittel habe ich nach den
Regeln wissenschaftlicher Praxis eindeutig gekennzeichnet. 
Mir ist bewusst, dass ich
fremde Texte und Textpassagen nicht als meine eigenen ausgeben darf und dass ein
Verstoß gegen diese Grundregel des wissenschaftlichen Arbeitens als Täuschungs- und
Betrugsversuch gilt, der entsprechende Konsequenzen nach sich zieht. Diese bestehen
in der Bewertung der Prüfungsleistung mit "nicht ausreichend" (5,0) sowie ggf. weiteren
Maßnahmen.

Außerdem bestätige ich, dass diese Arbeit in gleicher oder ähnlicher Form noch in keinem anderen Seminar vorgelegt wurde.
\vspace*{50pt}

Heidelberg, den \today \hspace{2cm} \underline{\phantom{Platz für die Unterschrift}}
\newpage
%%%%%%%%%%%%%%%%%%%%%%%%%%%%% </Antiplagiatserklärung> %%%%%%%%%%%%%%%%%%%%%%%%%%%%



%%%%%%%%%%%%%%%%%%%%%%%%%%%%%% <Inhaltsverzeichnis> %%%%%%%%%%%%%%%%%%%%%%%%%%%%%%%
% Table of contents
\tableofcontents
\newpage
%%%%%%%%%%%%%%%%%%%%%%%%%%%%%% </Inhaltsverzeichnis> %%%%%%%%%%%%%%%%%%%%%%%%%%%%%%





%%%%%%%%%%%%%%%%%%%%%%%%%%%%% <Hauptteil der Arbeit> %%%%%%%%%%%%%%%%%%%%%%%%%%%%%%
\pagenumbering{arabic}

%%%%%%%%%%%%%%%%%%%%%%%%%%%%% <Einleitung> %%%%%%%%%%%%%%%%%%%%%%%%%%%%%%%%%%%%%%%%
\section{Einleitung}
In letzter Zeit hat sich das Sortier- und Suchverhalten der Menschheit geändert. Sortierte man früher noch seine Dokumente in Ordner, ist man heute glücklich, wenn man mit leistungsstarken Suchalgorithmen schnell und präzise das gewünschte Dokument findet.\\
Dabei beschränken sich Suchanfragen nicht nur auf lokale Daten, sondern werden sogar großteils ans Web gestellt. Eine häufige Anfrageform an Suchmaschinen sind hierbei die \textit{named entity queries}\cite{paper:Guha}.\\
\begin{table}[h]
	\centering
	\begin{tabular}{c}
		\includegraphics[scale=.2]{pics/overview}
	\end{tabular}
	\caption{Häufige Suchanfrage an Suchmaschinen}
	\label{tab:freq_query}
\end{table}\\
Ein Problem bei solchen Suchanfragen ist es, die Häufigkeit der \textbf{named entity} in Dokumenten richtig zu erfassen\cite{paper:Na}.
Denn die Referenzierung des Objektes, das sich hinter einem Eigennamen verbirgt, wird sowohl mit dem Eigennamen selbst, als auch mit kompakteren Ausdrücken vorgenommen.\\
So wird beispielsweise in einem Text, der von Peter Chen handelt, die Person \textit{Peter Chen} auch mit \textit{"`er"'} referenziert.\\
\\
\fbox{ 
\begin{minipage}{12cm}
	"`\textbf{Chen} has recieved serveral awards in the fields of Information Technology. \textbf{He} received the Data Resource Management Technology Award [\ldots]"' \cite{wiki:peter_chen}
\end{minipage}
}\\
\\
In der folgenden Hausarbeit wird ein statistisches Verfahren erklärt, welches die Häufigkeit von Eigennamen in Dokumenten schätzen soll. Grundlage für diese Ausarbeitung ist ein Paper der Herren Na und Ng über "`A 2-Poisson Model for Probabilistic Coreference of Named Entities for Improved Text Retrieval"'\cite{paper:Na}.



%%%%%%%%%%%%%%%%%%%%%%%%%%%%% <Coreferentially Enhanced Entity Frequency (CEEF)> %%
\section{CEEF}
%%%%%%%%%%%%%%%%%%%%%%%%%%%%% <CEEF - Basic> %%%%%%%%%%%%%%%%%%%%%%%%%%%%%%%%%%%%%%
\subsection{Plausible Annahme über die Verteilung von Koreferenzen}
%PDF-Features von latex nachlesen; wir bräuchten einen link auf die Einleitung - und zwar auf die fbox
Die Häufigkeit von Eigennamen in Dokumenten ist schwer zu erfassen. Zählt man nur die \textit{named entites} selbst, erhält man die raw entity frequency:
\[tf\left( e;d \right)=\sharp \text{(named entity)}\]
Die raw entity frequency beachtet  noch keine zusätzlichen Koreferenzen auf das Objekt hinter dem gesuchten Eigennamen. Um später ein besseres Ranking ausführen zu können, möchten wir möglichst alle Referenzen erfassen. Also addieren wir die Häufigkeit der Koreferenzen zu unserer raw entity frequency:
\[tf\left( e;d \right)\leq tf_{true}\left( e;d \right)=tf\left( e;d \right)+\underbrace{atf(e_Q;A,d)}_{Korefernzen}\]
Theoretisch sind wir nun am Ziel. Praktisch ist es aber nicht so einfach, die Häufigkeit der Koreferenzen zu bestimmen. Mit Programmen, die Koreferenzen vollständig auflösen, geht ein überdimensionaler Berechnungsaufwand einher. Zudem sind diese Programme noch nicht besonders Präzise und ordnen weniger als 70\% der Koreferenzen richtig zu.\cite{paper:Na}
\\
Seung-Hoon Na und Hwee Tou Ng untersuchten deswegen einen statistischen Ansatz zur Schätzung der Koreferenzen. Dafür nehmen sie folgendes an:
\\
\fbox{ 
\begin{minipage}{12cm}
	``Our key assumption is that the frequency of anaphoric expressions is distributed over named entities in a document according to the probabilities of whether the document is elite for the named entities.''\cite{paper:Na}
\end{minipage}
}\\
\\
Ausgehend von dieser Annahme wird nun ein statistisches Modell erstellt, das in der Lage ist, die Häufigkeit von Eigennamen in Dokumenten zu schätzen.


%%%%%%%%%%%%%%%%%%%%%%%%%%%%% <CEEF - Formalisierung> %%%%%%%%%%%%%%%%%%%%%%%%%%%%%
\subsection{Formalisierung}
Bevor wir zur eigentlichen Statistik kommen, legen wir noch die mathematische Schreibweise für diese Aufgabe fest.
Zunächst sei Q unsere Query. Die Query besteht hier zur Vereinfachung aus einem einzigen Eigennamen.\\
Weiter sei e ein beliebiger Eigennamen. Entspricht e der Suchanfrage der Query Q, so nennen wir den Eigennamen gesuchte Entität $e_Q$. Ist e ein beliebiger nicht gesuchter Eigenname, so nennen wir die Entität nicht gesuchter Eigenname $e_N$.\\
Die Koreferenzen werden \textit{mögliche Koreferenzen} zu einem Eigennamen genannt, wenn sie semantisch zu dem Eigennamen passen. Unterteilen wir beispielsweise die Eigennamen in Objekte und Personen, so sind \{``he'',``she''\} mögliche Koreferenzen zu Personen. Eine mögliche Koreferenz auf Objekte ist \{``it''\}. Die Menge aller möglichen Koreferenzen wird A genannt.\\
%Eventuell Beispiel einfügen
Zählt man alle möglichen Koreferenzen A in einem Dokument d, so schreibt man $tf\left( A;d \right)$. Die Menge aller möglichen Eigennamen zu einer Menge von Koreferenzen in einem Dokument d wird mit $\varepsilon\left( A;d \right)$ beschrieben.


 \begin{itemize}
	\item $Q$ = query (Suche)
	\item $e_Q$ = query entity (gesuchte Entität)
	\item $e_N$ = non-query entity
	\item $A$ = Menge plausibler anaphorischer Ausdrücke
	\item $tf(A;d)$ = Anzahl von $A$ in Dokument $d$
	\item $\varepsilon (A;d)$ = Menge plausibler Entitäten in Dokument $d$
	\item $tf(e;d)$ = raw entity frequency
	\item $atf(e;A,d)$ = anaphoric entity frequency
  \end{itemize}

  Ausgerüstet mit einer einheitlichen Schreibweise wird zunächst Eliteness diskutiert und dann der Kern des Ganzen - das Poisson-Modell für statistische Häufigkeit von Koreferenzen - hergeleitet.



%%%%%%%%%%%%%%%%%%%%%%%%%%%%% <CEEF - Eliteness> %%%%%%%%%%%%%%%%%%%%%%%%%%%%%%%%%%%
\subsection{Eliteness}

\begin{defi}
	A document is \underline{elite} for a term if the document is ``about'' the concept represented by the term \cite{paper:Robertson}.
\end{defi}


Bei genauerer Begutachtung der Definition liegt die grundlegende Annahme von Seung-Hoon Na und Hwee Tou Ng, dass die Bezüge der anaphorischen Ausdrücke auf die Entitäten im Zusammenhang stehen, ob ein Dokument bezüglich dieser Entität „elite“ ist, oder nicht, nahe. Wir definieren uns also einen weiteren Faktor:

\begin{defi}
$P(\textbf{E}(e) = 1 | d)$ ist die Wahrscheinlichkeit, dass ein Dokument für eine Entität $e$ „elite“ ist.
\end{defi}

Jetzt können wir folgendes setzen:

\begin{equation}
P(e_Q | A,d) = \frac{P(\textbf{E}(e) = 1 | d)}{\sum_{e \in \varepsilon (A;d)} P(\textbf{E}(e) = 1 | d)}
\end{equation}
Wir machen also nichts anderes, als die Wahrscheinlichkeit der Eliteness des Dokumentes bezogen auf unsere gesuchte Entität in Relation mit den Wahrscheinlichkeiten der anderen Named Entities zu stellen.\\
Jetzt vereinfachen wir die Formel noch ein wenig, indem wir zuerst alle non-query Entitäten nach der Wahrscheinlichkeit der Eliteness sortieren. Danach werden die $K$ non-query Entitäten mit der höchsten Wahrscheinlichkeit selektiert, sodass sich unsere Formel auf Folgendes beschränkt:

\[ P(e_Q | A,d) \approx \frac{P(\textbf{E}(e) = 1 | d)}{P(\textbf{E}(e) = 1 | d) + \sum_{i=1}^{K} P(\textbf{E}(e_{n}^{(i)}) = 1 | d)} \]

$K$ ist in dem Fall frei gewählt und kann entsprechend angepasst werden. So dass tatsächlich nur Wahrscheinlichkeiten vernachlässigt werden, die nicht ins Gewicht fallen.\\
Nun suchen wir uns eine repräsentative non-query Entität heraus und setzen die Wahrscheinlichkeit der Eliteness aller anderen der Top-$K$ Entitäten mit der Wahrscheinlichkeit der Herausgesuchten gleich. Somit erhalten wir eine stark vereinfachte, aber trotzdem recht genaue Alternative von $(1)$.

\[ P(e_Q | A,d) \approx \frac{P(\textbf{E}(e) = 1 | d)}{P(\textbf{E}(e) = 1 | d) + K \cdot P(\textbf{E}(e_N) = 1 | d)} \]


Im Folgenden werden wir eine performante und einfache Methode beschreiben, mit der es möglich ist, den Faktor $P(\textbf{E}(e) = 1 | d)$ zu berechnen.




%%%%%%%%%%%%%%%%%%%%%%%%%%%%% <CEEF - Poisson> %%%%%%%%%%%%%%%%%%%%%%%%%%%%%%%%%%%%
\subsection{Poisson}
Aufgrund ihrer Wichtigkeit, nochmal die grundlegende Annahme von Seung-Hoon Na und Hwee Tou Ng:\\
\fbox{ 
\begin{minipage}{12cm}
	``Our key assumption is that the frequency of anaphoric expressions is distributed over named entities in a document according to the probabilities of whether the document is elite for the named entities.''\cite{paper:Na}
\end{minipage}
}\\
\\
Um diese Annahme in ein mathematisches Modell zu übersetzen, muss man die Wahrscheinlichkeit abschätzen, dass ein Dokument ``elite'' ist. Wir suchen also eine Formel für
\[P\left( E\left( e \right)=1|d \right)=\ ?\]
Zunächst interessiert uns hierfür die Wahrscheinlichkeit, dass eine beliebige Entität $tf$-fach in einem Dokument vorkommt. Dafür wird ein 2-Poisson Mixture Modell herangezogen.

\begin{equation}
P\left( tf \right)=\pi_e\frac{e^{-\lambda_e}\lambda_{e}^{ tf }}{tf!}+\left( 1-\pi_e \right)\frac{e^{-\mu_e} \mu_e^{tf}}{tf!}
\end{equation}
Der erste Term repräsentiert hierbei die Wahrscheinlichkeit des $tf$-fachen Auftretens der Entität in der Funktion als ``elite''-Eigennamen, der zweite Term steht für die Wahrscheinlichkeit des tf-fachen Auftretens der Entität als ``non-elite''-Eigennamen.
\[P\left( tf \right)=\underbrace{\pi_e\frac{e^{-\lambda_e}\lambda_{e}^{ tf }}{tf!}}_{Wahrscheinlichkeitsanteil - elite} +\underbrace{\left( 1-\pi_e \right)\frac{e^{-\mu_e} \mu_e^{tf}}{tf!}}_{Wahrscheinlichkeitsanteil - nonelite}\]
Dabei deutet $\pi_e$ auf den Erwartungswert hin, ob ein Dokument ``elite'' ist oder nicht.\\
$\pi_e$, $\lambda_e$ und $\mu_e$ werden festgelegt als:
\[\pi_e=\frac{df\left( e \right)}{df\left( e \right)+N}, \ \lambda_e=\frac{cf\left( e \right)}{df\left( e \right)}, \ \mu_e=\frac{cf(e)}{N}\]
wobei
\begin{itemize}
	
	\item $N$: Anzahl der Dokumente
	\item $df(e)$: Anzahl der Dokumente, in denen $e$ vorkommt
	\item $cf(e)$: Anzahl von $e$ in allen Dokumenten
\end{itemize}
Damit haben wir ein Modell, dass die Wahrscheinlichkeit der Häufigkeit von Entitäten in Dokumenten schätzt. Wichtig ist hierbei, dass sowohl das Auftreten von ``elite''-Entitäten in Dokumenten mit einer Poissonverteilung modelliert wird, als auch das ``zufällige'' Auftreten der ``nonelite''-Entitäten.\\
Eine einfache Poissonverteilung sieht in Formel und im Diagramm wie folgt aus:
\[P_\lambda(k)=\frac{\lambda^k}{k!}e^{-\lambda}\]
\begin{figure}[h]
	\centering
	\begin{tabular}{c}
		\includegraphics[scale=0.5]{pics/poisson_basic}
	\end{tabular}
	\caption{Einfache Poissonverteilung}
	\label{tab:poisson_basic}
\end{figure}

Mit Hilfe von $(2)$ können wir nun $P(\textbf{E}(e)=1|d)$ bestimmen. Wir setzen
\[ P(\textbf{E}(e)=1|d) = \frac{\pi_e P(tf(e;d)|\textbf{E}(e)=1)}{\pi_e P(tf(e;d)|\textbf{E}(e)=1) + (1-\pi_e) P(tf(e;d)|\textbf{E}(e)=0)} \]
mit
\begin{eqnarray*}
	P(tf(e;d)|\textbf{E}(e)=1) &=& \frac{e^{-\lambda_e}\lambda_e^{tf}}{tf!}\\
	P(tf(e;d)|\textbf{E}(e)=0) &=& \frac{e^{-\mu_e}\mu_e^{tf}}{tf!}
\end{eqnarray*}

und erhalten so

\begin{eqnarray*}
	P(\textbf{E}(e)=1|d) 	&=& 	\frac{\pi_e P(tf(e;d)|\textbf{E}(e)=1)}{\pi_e P(tf(e;d)|\textbf{E}(e)=1) + (1-\pi_e) P(tf(e;d)|\textbf{E}(e)=0)} \\
	&=&	\frac{\pi_e \frac{e^{-\lambda_e}\lambda_e^{tf}}{tf!}}{\pi_e \frac{e^{-\lambda_e}\lambda_e^{tf}}{tf!} + (1 - \pi_e) \frac{e^{-\mu_e}\mu_e^{tf}}{tf!}}\\
	&=&	\frac{\pi_e{e^{-\lambda_e}\lambda_e^{tf}}}{\pi_e {e^{-\lambda_e}\lambda_e^{tf}} + (1 - \pi_e) e^{-\mu_e}\mu_e^{tf}}\\
	&=&	\frac{\pi_e}{\pi_e + (1-\pi_e)\frac{e^{-\mu_e}\mu_e^{tf}}{e^{-\lambda_e}\lambda_e^{tf}}}\\
	&=&	\frac{\pi_e}{\pi_e + (1-\pi_e)e^{\lambda_e - \mu_e}\left( \frac{\mu_e}{\lambda_e}\right)^{tf(e;d)}}	
\end{eqnarray*}

Jetzt werden 2 Fälle unterschieden:
\begin{enumerate}
	\item $e=e_Q$ \\
		Wir machen die plausible Annahme, dass $tf(e;d) \geq 1$, wenn das Dokument $d$ elite ist für die gesuchte Entität $e$. Alle anderen Dokumente im Test seien non-elite bezüglich $e$. \\
		Dann können wir die $\pi_e, \lambda_e \text{ und } \mu_e$ wie oben beschrieben ersetzen und erhalten
		\begin{eqnarray*}
			P(\textbf{E}(e_Q)=1|d)	&=&	\frac{\pi_e}{\pi_e + (1-\pi_e)e^{\lambda_e - \mu_e}\left( \frac{\mu_e}{\lambda_e}\right)^{tf(e;d)}}\\
			&=&	\frac{\frac{df(e)}{df(e) + N}}{\frac{df(e)}{df(e) + N} + \left(1-\frac{df(e)}{df(e) + N}\right) e^{\frac{cf(e)}{df(e)}-\frac{cf(e)}{N}} \left( \frac{\frac{cf(e)}{N}}{\frac{cf(e)}{df(e)}}\right)^{tf(e;d)}}\\
%			&=&	\frac{1}{1 + \frac{e^{\frac{cf(e)}{df(e)}-\frac{cf(e)}{N}}\left( \frac{df(e)}{N}\right)^{tf(e;d)}}{\frac{df(e)}{df(e) + N}} - e^{\frac{cf(e)}{df(e)}-\frac{cf(e)}{N}}\left( \frac{df(e)}{N}\right)^{tf(e;d)} }\\
			&=&	\frac{1}{1+ \left( \frac{df(e)}{N}\right)^{tf(e;d)-1}e^{\frac{cf(e)}{df(e)}-\frac{cf(e)}{N}}}
		\end{eqnarray*}

	\item $e \neq e_Q$ \\
		Auch hier machen wir zu Beginn zwei Annahmen:
		\begin{enumerate}
			\item Es seien alle non-query Entitäten vom selben Typ, wie die query Entität
			\item Es sei die Termfrequency ($tf(e;d)$) einer Top-$K$ non-query Entität gleich dem Durschnittsvorkommen einer query Entität in einem elite Dokument
		\end{enumerate}
		Unter diesen Vorraussetzungen dürfen die $\lambda_e, \mu_e \text{ und } \pi_e$ wie oben gewählt werden. Dies führt uns erneut zu der Formel
		\[ 	P(\textbf{E}(e)=1|d) = \frac{1}{1+ \left( \frac{df(e)}{N}\right)^{tf(e;d)-1}e^{\frac{cf(e)}{df(e)}-\frac{cf(e)}{N}}} \]
		Aufgrund von (b) gilt jetzt aber, dass $tf(e;d) = \frac{cf(e)}{df(e)}$, weshalb wir in diesem Fall zu dem Ergebnis kommen, dass
		\[	P(\textbf{E}(e_N)=1|d) = \frac{1}{1+ \left( \frac{df(e)}{N}\right)^{\frac{cf(e)}{df(e)}-1}e^{\frac{cf(e)}{df(e)}-\frac{cf(e)}{N}}} \]
\end{enumerate}

Mit diesen Ergebnissen können wir jetzt unseren Wahrscheinlichkeitsfaktor 
\[P(e_Q | A,d) \approx \frac{P(\textbf{E}(e) = 1 | d)}{P(\textbf{E}(e) = 1 | d) + K \cdot P(\textbf{E}(e_N) = 1 | d)} \]
abschätzen und damit die gesuchte wahre Termfrequency
\[ tf_{true}(e_Q;d) \approx tf(e_Q;d) + P(e_Q|a,d)tf(A;d) \]
Wir wollen zum Schluss noch darauf aufmerksam machen, dass es sinnvoller ist, statt der normalen Termfrequency, die so genannte „normalized Termfrequency“ zu verwenden.\\
\begin{defi}[normalized Termfrequency]
\[ ntf(e;d) = \frac{tf(e;d) \cdot avglen}{len(d)} \]
Wobei $avglen$ die Durchschnittslänge eines Dokuments im Test ist und $len(d)$ die Länge des aktuellen Dokuments.
\end{defi}
Die normalized Termfrequency setzt also das Vorkommen der Entitäten in Relation mit der Länge der Dokumente. Das ist wichtig, da die Termfrequency in langen Dokumenten ja meist höher ist als in Kurzen, dies aber das Dokument nicht unbedingt relevanter macht.
Somit setzen wir also
\[ ntf_{true}(e_Q;d) \approx ntf(e_Q;d) + P(e_Q|a,d)ntf(A;d) \]
und sind fertig.


%%%%%%%%%%%%%%%%%%%%%%%%%%%%% <Zusätzliche Schritte> %%%%%%%%%%%%%%%%%%%%%%%%%%%%%%
\section{Zusätzliche Bearbeitungsschritte}

%%%%%%%%%%%%%%%%%%%%%%%%%%%%% <Automatische Klassifizierung der NE> %%%%%%%%%%%%%%%
\subsection{Automatische Klassifzierung der named entities}


%%%%%%%%%%%%%%%%%%%%%%%%%%%%% <Feedback-Based Identification> %%%%%%%%%%%%%%%%%%%%%
\subsection{Automatische Erkennung von Koreferenzen}
Da die Menge an Koreferenzen 
\[A_O=\left\{\text{'it', 'its'}\right\}\]
für Objekte nicht ausreichend ist, haben die Autoren des Papers \NaNg\cite{paper:Na} ein Verfahren entwickelt, um die Menge $A_O$ zu erweitern. Sie nennen ihren Algorithmus ``Feedback-Based Identification of Anaphoric Expressions''.



%%%%%%%%%%%%%%%%%%%%%%%%%%%%% <Retrieval Method> %%%%%%%%%%%%%%%%%%%%%%%%%%%%%%%%%%
\subsection{Retrieval Method}

Um unsere Methode zu testen, müssen wir uns zunächst auf eine retrieval method bzw. Ranking Methode festlegen. Die Herren Na und Ng verwenden hierfür eine Variante des BM25-Models, welche die CEEF-Methode zur Bestimmung der Termfrequency benutzt und wie folgt definiert ist:
\[ BM25_{CEEF}(e_Q;d) = OkapiTF(e_Q;d) \cdot IDF(e_Q) \]

wobei

\[ OkapiTF(e_Q;d) = \frac{ (k_1 + 1) \cdot tf_{CEEF}(e_Q;d)}{tf_{CEEF}(e_Q;d) + k_1\left( (1-b) + b \frac{len(d)}{avglen} \right)} \]

$b$ und $k_1$ dienen hier der Normierung. $IDF(e_Q)$ steht für die inverse Documentfrequency und wird vom BM25-Model zur Gewichtung verwendet und wird wie folgt berechnet:
\[ IDF(e_Q) = \log \left( \frac{N}{df(e_Q)} \right) \text{\cite{wiki:tf_idf}} \]

Es ist zudem noch zu erwähnen, dass diese Methode mit anderen aktuellen und als „state-of-the-art“ geltenden retrieval methods kombinierbar ist.\\
Dies funktioniert so, dass man sich einen Normierungfaktor $\alpha \in [0,1]$ wählt und damit die zwei Methoden entsprechend gewichtet und aufaddiert. Man nennt dieses Verfahren „Lineare Interpolation“ \cite{paper:Vogt}. 

\[ Sim(Q;d) = (1-\alpha) \cdot Sim_{BASE}(Q;d) + \alpha \cdot BM25_{CEEF}(e_Q;d) \]

Hier steht $Sim_{BASE}(Q;d)$ für ein beliebiges state-of-the-art Verfahren, zum Beispiel das Language Modeling \cite{paper:Zhai1}\cite{paper:Zhai2}.



% References (Literaturverzeichnis):
% a) Style (with abbreviations: use alpha):
\bibliographystyle{plainnat-d}
% b) The File:
\newpage
\bibliography{seminararbeit}

\end{document}
\end{document}
\end{document}
