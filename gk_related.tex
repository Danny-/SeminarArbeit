\subsection{Arbeiten zu diesem Thema}

Die Idee, Koreferenzen beim text retrieval mit einzubeziehen, ist nicht neu. 1990 erforschte E.D. Liddy in seiner Arbeit „Anaphora in ratural language processing and information retrieval“ \cite{paper:Liddy} dieses Themengebiet anhand eines kleinen Korpus von 487 Dokumenten. Die anaphorischen Ausdrücke wurden hier von Hand den entsprechenden Eigennamen zugeordnet. Das Ergebnis hierbei war, dass eine Besserung des Ergebnisses, nur in ein paar wenigen Anfragen zu erkennen war.\\
1996 veröffentlichten A. Pirkola und K. Järvelin ihr Paper „The effect of anaphor and ellipsis resolution on proximity searching in a text database“ \cite{paper:Pirkola}. Sie zeigten, dass die Auflösung von anaphorischen Ausdrücken sehr wohl eine starke, positive Auswirkung, auf das Ergebnis hat, da in ihren Tests viel mehr, für die entsprechenden Suchanfragen, relevante Dokumente gefunden wurden. Es wurden hier 55.000 finnische Zeitungen als Korpus verwendet und auch hier wurden die Anaphern per Hand zugeordnet. Man kann ihre Arbeit als die erste Erfolgreiche in diesem Gebiet bezeichnen. Allerding muss man beachten, dass der Korpus noch sehr klein und keinerlei automatisierung in ihrem Vorgehen vorhanden war.\\
Erst 2003 wurde eine Arbeit veröffentlicht, die genau dieses Problem behandelte. R.J. Edens, H.L. Gaylard, G.J.F. Jones and A.M. Lam-Adesina entwickelten einen Algorithmus, mit dessen Hilfe anaphorische Ausdrücke ihren Eigennamen zuordnen konnte \cite{paper:Edens}. Sie testeten ihr System an der CLEF 2002 English Kollektion \cite{paper:Braschler}, welche 100.000 Dokumente und 50 verschiedene Themen enthält. Ihr Ergebnis war, dass durch automatisierte Auflösung die Performance des Retrievals stark verbessert werden konnte.
